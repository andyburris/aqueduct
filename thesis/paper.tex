\documentclass[manuscript,review,anonymous]{acmart}
% note: Overleaf includes acmart.cls for you so you don't even need that
% it even includes the ACM-ReferenceFormat.* files, but I edited the .bst file to not warn about missing publisher/address for inproceedings

\setcopyright{acmcopyright}
\copyrightyear{2022}
\acmYear{2022}
%\acmDOI{10.1145/1122445.1122456}

\acmConference[Woodstock '18]{Woodstock '18: ACM Symposium on Neural
  Gaze Detection}{June 03--05, 2018}{Woodstock, NY}
\acmBooktitle{Woodstock '18: ACM Symposium on Neural Gaze Detection,
  June 03--05, 2018, Woodstock, NY}
\acmPrice{15.00}
\acmISBN{978-1-4503-XXXX-X/18/06}

\usepackage{booktabs} % use booktabs instead of ugly regular tables
\usepackage{graphicx} % more figure options
\RequirePackage[l2tabu, orthodox]{nag} % checks for common LaTeX errors
\usepackage{microtype} % better typesetting
\usepackage[utf8]{inputenc} % lenient to utf-8 characters like smart quotes
\usepackage{refcheck} % warns about unreferenced figures/tables that have labels. remove it before submitting
\usepackage[subtle]{savetrees} % denser formatting, you can comment this out

\interfootnotelinepenalty=10000 % split footnotes are ugly
\tolerance=400 % reduce how often words stick out into columns at the expense of word spacing

\graphicspath{{figures/}} % put all your figures in this folder

\begin{document}

\title{Catchy Name: Descriptive Title for Your Paper}

%% "authornote" and "authornotemark" commands
%% used to denote shared contribution to the research.
\author{First Last}
\affiliation{%
  \institution{Brown University}
  \country{United States}
}
\email{youremail@example.com}

\renewcommand{\shortauthors}{Lastname, et al.}

\begin{abstract}
This is not an introduction.
No need to spend multiple sentences setting the context or motivation.
Avoid going on and on about the work in general.

Avoid saying what this paper will talk about (e.g. we will report the results of the user study).
Say the things, don't say that the paper will say the things.
Just summarize the work.

Answer these questions concisely and you will be okay.
What problem did you investigate (2 sentences)?
What did you do (2 sentences)?
What were the key findings (2 sentences)?
What is the larger implication (1 sentence)?

After you write your abstract, consider starting it at the third sentence to skip the fluff.
\end{abstract}

%%
%% The code below is generated by the tool at http://dl.acm.org/ccs.cfm.
%% Please copy and paste the code instead of the example below.
%%
\begin{CCSXML}
<ccs2012>
 <concept>
  <concept_id>10010520.10010553.10010562</concept_id>
  <concept_desc>Computer systems organization~Embedded systems</concept_desc>
  <concept_significance>500</concept_significance>
 </concept>
 <concept>
  <concept_id>10010520.10010575.10010755</concept_id>
  <concept_desc>Computer systems organization~Redundancy</concept_desc>
  <concept_significance>300</concept_significance>
 </concept>
 <concept>
  <concept_id>10010520.10010553.10010554</concept_id>
  <concept_desc>Computer systems organization~Robotics</concept_desc>
  <concept_significance>100</concept_significance>
 </concept>
 <concept>
  <concept_id>10003033.10003083.10003095</concept_id>
  <concept_desc>Networks~Network reliability</concept_desc>
  <concept_significance>100</concept_significance>
 </concept>
</ccs2012>
\end{CCSXML}

\ccsdesc[500]{Computer systems organization~Embedded systems}
\ccsdesc[300]{Computer systems organization~Redundancy}
\ccsdesc{Computer systems organization~Robotics}
\ccsdesc[100]{Networks~Network reliability}

\keywords{datasets, neural networks, gaze detection, text tagging}

\maketitle

\section{Introduction}

Give a brief overview of the problem space.
Spend at least 10 minutes crafting an attention-grabbing interesting first sentence.
A famous quote is okay if it is highly relevant.

Motivate the reader.
Why is this important?
Why did you work on this?

Why is the current state-of-the-art insufficient?
What did you do? (A common mistake is to only talk about the benefits of what you did, without ever saying in plain words what you did)
How does your work extend or differ from the generally accepted approaches?

What are the research hypotheses or questions you are exploring?

What are the main contributions that you make in your paper?
Contributions are contributions to the field that others will appreciate.
The fact that you made a system or coined a term is not a contribution in itself.
Is there a primary contribution and a secondary contribution?
Use bullet points if there are more than two.
If you have more than two or three contributions, you're saying that you have impressively packed multiple papers worth of work in it.

\section{Related Work}
What the the several areas of research that relate to you work?
Give a short description here of the related areas.

\subsection{Related Area 1}
This should be the most closely related area.

What are the papers that are most relevant? A good rule of them is to have at least 3 papers from that conference in your references (if you can't even find a few related papers published there, then why would they accept something so unrelated to what they publish?). If you're submitting to a mainstream HCI conference like CHI/UIST/IMWUT/CSCW then most of the references should be mainstream HCI papers, not domain-specific journals or venues that HCI researchers have not heard of; not a rule, but more like a rule of thumb, so if you're finding very few HCI related papers, you may want to rescope your Google Scholar search.
What did those paper find? (note, talk more about the findings, and less about the method or what the authors did)
Give credit where it is due.
Do not dismiss or disparage other peoples' work here.
How is your work different or (even better) how does your work build off this existing work (2 sentence max)?

\subsection{Related Area 2}
Another related area.
Same idea.

Example citation~\cite{wallace2017drafty}. If you're really ambitious, use autoref (requires hyperref package). Note the tilde shown in the source is a non-breaking space. It's a space, so you don't need to put an additional space before it.

Some reviewers don't like it if you use citations as part of a sentence, like saying that~\cite{wallace2017drafty} is a relevant paper. Rather, they want the author name(s) there as well, like Wallace et al.~\cite{wallace2017drafty} is a relevant paper.

\section{System}
This section is for if you built a system or developed algorithms before doing an evaluation.
Consider that the reader just thought about what you just said in Related Work, and smoothly transition them to thinking about the work in this paper.
Don't abruptly jump to your system like it's a completely unrelated thing.

Change the name of this section as appropriate; for example, ``The GruffusGruffus System" or ``User WoozleWazzle Prediction".

Strongly consider a subsection here about Design Considerations, or if that is a key part of the work, put it in a new section before this one. having this sections reduces the jump when going from Related Work to a bunch of technical description of the system, which doesn't say anything about \emph{why} the system was built that way.

What did the system/algorithm do?
How did your design your system/algorithm? Don't write at the level where you are describing exactly the programming language, the software libraries, etc. write one level above, and the way to do that is write as if someone will be implementing it in a completely different software stack.
Why did you make the design decisions you did?

Say that the system is or will be open source, if that's true. Link to the URL of the potential open source page where it will be put, but anonymize it for submission.

If there are multiple important components to the system, or both an algorithm and a system, feel free to split this section into multiple sections.

\section{Method}
Either have a Data subsection if the data was crawled online or a Procedure subsection if you did a user study.
You probably don't want both subsections so delete the other.

If this paper has a substantial qualitative component, describe the framework you used (Grounded Theory, Thematic Analysis, etc.).

If you went through IRB review, say that the study was reviewed by our institution's Human Subjects Board.

\subsection{Data}
Where the data is from?
How did you get it?
What users are represented?
What do you know about their demographics (at least age ranges, gender, where they are from, occupation)?
How much data do you have?

Feel free to provide summary statistics of your data here.

\subsection{Study Procedure}
How did you run your study?
Why did you use this procedure, run the study this way, how did you consider ordering effects?
How did you recruit participants?
What were their demographics (at least age ranges, gender, where they are from, occupation)?
What did the participants do?
What did you ask them?
How long did the study take?
What things did you keep in mind (maybe behaviors you tried to observe or things you avoided telling participants) during the study?
Were there any problems (it's okay, but report if data couldn't be collected, a participant left during the study, or the participant couldn't perform the task)?

\subsection{Other}
This subsection is for any additional methods you used.
This could be a special method you used for analyzing the data; for example, grounded theory or an out-of-the-box machine learning library.
This could also be any preparation of the data you did, like asking experts to label the data (if you did have experts label the data, report the inter-rater reliability).
This could also be any experiment you ran to narrow down a particular hypothesis.

\section{Results}
You can have as many sections about results as you'd like.

Qualitative papers will have more topics.
One for each theme.

Quantitative papers should have at least one: Results or Findings.
If there is a single important result, just call the section Results and build up to this result.
If there are multiple interesting analyses, each with interesting findings, separate the results into multiple sections instead.

Support your results with tables and figures.
Be sure the paper makes sense if the reader completely ignores all the figures and captions.
Be sure the paper makes sense if the reader only looks at the figures and captions.
Figures should be in color, but still be just as comprehendable if printed in black and white.
Captions should state what the reader should see from the figure, i.e. the main point of the figure ``A causes B, C is 50\% larger than D'', not ``The figure shows how A and C relate to B and D''.
LaTeX sometimes lays out figures/text in a funny way which can be easily fixed
Don't worry about layout of text/figures on pages until the very end (after everyone has reviewed the paper and made final changes), let the figure float (don't force [h] or [t] on your figures until the day of submission)
Figures should be in vector format (e.g. .pdf, .svg) except for screenshots and photos
Figures should be trimmed (the edges of it should touch actual content in the figure) so that there's no extra whitespace
png files should be losslessly compressed
The font size in a figure should be about the same as the font size of your text. It's easy to make it too small, because you often zoom into figures to work on them, and forget that they will be printed on letter-size paper, making the text unreadable. This is probably the number one reviewer criticism.
Charts should be relatively straightforward (bar charts, scatter plot, line plots). Anything else requires extra work from the reviewers to interpret. Overly fancy charts are like the WordArt of high school papers.
Columns in tables should be right-aligned if they contain numbers, and left-aligned in they contain text. Don't use too much more decimal digits than are significant.

\begin{figure}
  \centering
    \includegraphics[width=1\linewidth]{example-image-a}
    \caption{Caption that makes sense even if just read separately from the paper. Prefer to describe the main point in the figure (e.g. "A is increasing over time"), rather than give a generic heading (e.g. "Chart of performance on the y-axis and time on the x-axis."}
    \label{fig:example_figure}
\end{figure}

Make sure your results include answers to your research question.
If not, either change your research question or do more experiments/studies.

Other possible things to answer:
Was there an improvement over the baseline(s)?
How did the independent variables affect the dependent variables?
Were there differences between different groups in your study?
What themes have emerged from your results?
Were there any unexpected findings?

Use appropriate statistical tests.
Statistical tests should be reported like icing on a cake: they are not the main feature, so they don't go first, and pure statistical testing values shouldn't be in a table or paragraph, because it'd be like eating icing on a plate.
But do not spend weeks learning to use the most obscure statistical tests.

\section{Discussion}
What do our findings mean?

How do the findings relate to the related work and existing theories?

What applications or tasks are possible now that we have made these discoveries/inventions?

What are some implications for designing systems/interfaces in the future?

Are there any key limitations to your work?

What are some areas of future work that are beyond the scope of this paper but are now possible?

\section{Conclusion}
What did you do?
What did you find?

What are the broader implications of your work?

\section{Acknowledgments}

Ask Jeff if there are any funding supporters to thank.

Thank anybody else you'd like: anyone who didn't contribute enough to be a co-author, people who gave good advice, someone who helped set up a piece of equipment, maybe even participants.

\bibliographystyle{ACM-Reference-Format}
% Make sure your references are not too verbose (see one of Jeff's papers for examples)
\bibliography{paper}
\end{document}
